В этом разделе описаны только вариативные параметры запросов, с учётом того что форматы запросов и ответов описаны в \ref{subsec:common_format}, параметры email и apikey для всех типов запросов одинаковы.

\subsection{Создание/удаление канала}

\begin{lstlisting}[language=json,firstnumber=1]
{
  "module": "channels",
  "action": "create"/"delete",
  "http": [<address1>, <address2>, ...],
  "mqtt": [<queue1>, <queue2>, ...]
}
\end{lstlisting}

При этом на сервере RabbitMQ создаются очереди с названиями <email>\textunderscore queue1, <email>\textunderscore queue2 и.т.д. В дальнейшем к этим очередям можно присоединяться, используя в качестве логина email а в качестве пароля apikey для подключения к RabbitMQ.

\subsection{Подписка/отписка канала на сообщения от модема}

Для того, чтобы направить все приходящие от модема или батча модемов сообщения в канал (подписаться на модем), или отписать канал от модема (батча модемов), нужно выполнить запрос вида:

\begin{lstlisting}[language=json,firstnumber=1]
{
  "module": "subscription",
  "action": "subscribe"/"unsubscribe",
  "http": {
    <address1>: {
      "modems": [<modem1>, <modem2> ...],
      "batches": [<batch1>, <batch2> ...],
    }
    ...
  },
  "mqtt": {
    <queue1>: {
      "modems": [<modem3>, <modem2> ...],
      "batches": [<batch1>, <batch5> ...],
    }
    ...
  }
}
\end{lstlisting}


\subsection{Получение списка каналов и привязанных к ним модемов для данного юзера}

Для получения списка каналов и модемов, на которые они подписаны, нужно выполнить запрос:

\begin{lstlisting}[language=json,firstnumber=1]
{
  "module": "info"
}
\end{lstlisting}

\subsection{Отправка сообщения модему}

Для отправки сообщения на модем нужно выполнить запрос вида:

\begin{lstlisting}[language=json,firstnumber=1]
{
  "module": "downlink",
  "modem_id": <modem_id>,
  "payload": <payload>
}
\end{lstlisting}

где payload - данные передаваемые на модем, в формате шестнадцатеричной строки (до 256 байт), а modem\textunderscore id - номер модема.